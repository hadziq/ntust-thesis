


% 除非校方修改了論文格式 (margins, header, footer, 浮水印)
% 或者需要增加所用的 LaTeX 套件,
% 或者要改預設中文字型、編碼
% 否則毋須修改本檔內容
% 論文撰寫,請修改以 my_  開頭檔名的各檔案




\usepackage[nospace]{cite}  % for smart citation
\usepackage{geometry}  % for easy margin settings
%\usepackage{subfigure}  % for subfigure  因為有衝突
%\usepackage[dvipdfm]{graphicx}  % for graphic   using eps
\usepackage[xetex]{graphicx}
%\usepackage{graphicx}  % for graphic   using eps

%\usepackage{epstopdf} % 當使用pdflatex時打開,如使用latex則不需開啟,此功能為將xxx.eps 自動判讀為XXX.pdf


%\usepackage{algorithmic}  %演算法使用
\usepackage{algorithm}
\usepackage{algorithmicx}
\usepackage{algpseudocode}

%
% margins setting
\geometry{verbose,a4paper,tmargin=3.5cm,bmargin=2cm,lmargin=3cm,rmargin=3cm}
%
\usepackage{amsmath} % 各式 AMS 數學功能
%\usepackage{amssymb} % 各式 AMS 數學符號 (如果amssymb和Xunicode產生衝突,請將amssymb移到最前面)
\usepackage{mathrsfs} %草寫體數學符號,在數學模式裡用 \mathscr{E} 得草寫 E
\usepackage{listings} % 程式列表套件
%
% listing setting
\lstset{breaklines=true,% 過長的程式行可斷行
extendedchars=false,% 中文處理不需要 extendedchars
texcl=true,% 中文註解需要有 TeX 處理過的 comment line, 所以設成 true
comment=[l]\%\%,% 以雙「百分號」做為程式中文註解的起頭標記,配合 MATLAB
basicstyle=\small,% 小號字體, 約 10 pt 大小
commentstyle=\upshape,% 預設是斜體字,會影響註解裏的英文,改用正體
%language=Octave % 會將一些 octave 指令以粗體顯示
}

\usepackage{url} % 在文稿中引用網址,可以用 \url{http://www.ntust.edu.tw} 方式

% 插圖套件 graphicx
% 使用者工作流程是用 pdftex 還是 latex + dvipdfmx?
% 視情況而有不同的參數
% 這裡作自動判斷
% 參考自
% http://www.tex.ac.uk/cgi-bin/texfaq2html?label=ifpdf
%\newcommand\mydvipdfmxflow{dvipdfmx}
%\newcommand\mypdftexflow{pdftex}
%
%\ifx\pdfoutput\undefined
%  % not running pdftex
%  \usepackage[dvipdfm]{graphicx}
%  \newcommand\myworkflow{dvipdfmx}  % set the flag for hyperref
%\else
%  \ifx\pdfoutput\relax
%    % not running pdftex
%    \usepackage[dvipdfm]{graphicx}
%    \newcommand\myworkflow{dvipdfmx}  % set the flag
%  \else
%    % running pdftex, with...
%    \ifnum\pdfoutput>0
%      % ... PDF output
%      \usepackage[pdftex]{graphicx}
%      \newcommand\myworkflow{pdftex}  % set the flag
%    \else
%      %...DVI output
%      \usepackage[dvipdfm]{graphicx}
%      \newcommand\myworkflow{dvipdfmx}  % set the flag
%    \fi
%  \fi
%\fi

\usepackage{fancyhdr}  % 借用增強功能型 header 套件來擺放浮水印 
% (佔用了 central header)
% 不需要浮水印的使用者仍可利用此套件,產生所需的 header, footer
%
% 啟動 fancy header/footer 套件
\pagestyle{fancy}
\fancyhead{}  % reset left, central, right header to empty
\fancyfoot[C]{\thepage} %中間 footer 擺放頁碼
\renewcommand{\headrulewidth}{0pt} % header 的直線; 0pt 則無線

% 如果不需要任何浮水印,則請把下列介於 >>> 與 <<< 之間
% 的文字行關掉 (行首加上百分號)
%% 浮水印 >>> 
\input{watermark/ntust_watermark.tex}
%% <<< 浮水印



% global page layout
\newcommand{\mybaselinestretch}{1.5}  %行距 1.5 倍 + 20%, (約為 double space)
\renewcommand{\baselinestretch}{\mybaselinestretch}  % 論文行距預設值
\parskip=2ex  % 段落之間的間隔為兩個 x 的高度
\parindent = 24Pt  % 段首內縮由 CJK 控制,所以這裡就設成不內縮

%%%%%%%%%%%%%%%%%%%%%%%%%%%%%
%  end of preamble
%%%%%%%%%%%%%%%%%%%%%%%%%%%%%