% Unless school modifies the thesis format (margins, header, footer, watermark)
% or needs to improve the use of LaTeX packages,
% or needs to change the default fonts or the encoding,
% then no need to modify the content inside this file.
% For thesis writing personalization, please modify the files starting with 'my_'

\usepackage[nospace]{cite}  		% for smart citation
\usepackage{geometry}  				% for easy margin settings
%\usepackage{subfigure}  			% for subfigure
%\usepackage[dvipdfm]{graphicx}  	% for graphic using eps
\usepackage[xetex]{graphicx}
%\usepackage{graphicx}  			% for graphic using eps

%\usepackage{epstopdf} 				% for automatic transformation (eps to pdf) while using pdflatex


%\usepackage{algorithmic}  			% for algorithm
\usepackage{algorithm}
\usepackage{algorithmicx}
\usepackage{algpseudocode}

%
% margins setting
\geometry{verbose,a4paper,tmargin=3.5cm,bmargin=2cm,lmargin=3cm,rmargin=3cm}
%
\usepackage{amsmath} 				% for mathematical function (AMS Format)
%\usepackage{amssymb} 				% for mathematical symbol (AMS Format)
										% if 'amssymb' and 'Xunicode' conflict, please put 'amssymb' beforehand
\usepackage{mathrsfs} 				% for cursive mathematical symbol,
										% for example, typing '\mathscr{E} may result 'cursive E'
\usepackage{listings} 				% for listing any programming code
%
% listing setting
\lstset{breaklines=true,			% a long line of code can be broken into lines of code
extendedchars=false,				
texcl=true,							
comment=[l]\%\%,					
basicstyle=\small,					% approximately 10 pt
commentstyle=\upshape,				% the default is 'italics', so set it upright
%language=Octave 					% some 'octave' instruction will appear in bold
}

\usepackage{url} 					% for trackbacking URL in the document; usage: \url{http://www.ntust.edu.tw}

% 插圖套件 graphicx
% 使用者工作流程是用 pdftex 還是 latex + dvipdfmx?
% 視情況而有不同的參數
% 這裡作自動判斷
% 參考自
% http://www.tex.ac.uk/cgi-bin/texfaq2html?label=ifpdf
%\newcommand\mydvipdfmxflow{dvipdfmx}
%\newcommand\mypdftexflow{pdftex}
%
%\ifx\pdfoutput\undefined
%  % not running pdftex
%  \usepackage[dvipdfm]{graphicx}
%  \newcommand\myworkflow{dvipdfmx}  % set the flag for hyperref
%\else
%  \ifx\pdfoutput\relax
%    % not running pdftex
%    \usepackage[dvipdfm]{graphicx}
%    \newcommand\myworkflow{dvipdfmx}  % set the flag
%  \else
%    % running pdftex, with...
%    \ifnum\pdfoutput>0
%      % ... PDF output
%      \usepackage[pdftex]{graphicx}
%      \newcommand\myworkflow{pdftex}  % set the flag
%    \else
%      %...DVI output
%      \usepackage[dvipdfm]{graphicx}
%      \newcommand\myworkflow{dvipdfmx}  % set the flag
%    \fi
%  \fi
%\fi

\usepackage{fancyhdr}  % 借用增強功能型 header 套件來擺放浮水印 
% (佔用了 central header)
% 不需要浮水印的使用者仍可利用此套件,產生所需的 header, footer
%
% 啟動 fancy header/footer 套件
\pagestyle{fancy}
\fancyhead{}  % reset left, central, right header to empty
\fancyfoot[C]{\thepage} %中間 footer 擺放頁碼
\renewcommand{\headrulewidth}{0pt} % header 的直線; 0pt 則無線

% 如果不需要任何浮水印,則請把下列介於 >>> 與 <<< 之間
% 的文字行關掉 (行首加上百分號)
%% 浮水印 >>> 
\input{watermark/ntust_watermark.tex}
%% <<< 浮水印



% global page layout
\newcommand{\mybaselinestretch}{1.5}  %行距 1.5 倍 + 20%, (約為 double space)
\renewcommand{\baselinestretch}{\mybaselinestretch}  % 論文行距預設值
\parskip=2ex  % 段落之間的間隔為兩個 x 的高度
\parindent = 24Pt  % 段首內縮由 CJK 控制,所以這裡就設成不內縮

%%%%%%%%%%%%%%%%%%%%%%%%%%%%%
%  end of preamble
%%%%%%%%%%%%%%%%%%%%%%%%%%%%%